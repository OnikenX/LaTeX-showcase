\documentclass[a4paper]{article}
\usepackage[utf8]{inputenc}
\usepackage{xspace}
\usepackage[portuguese]{babel}

%Fontes especias de matemática
\usepackage{amsfonts}

%hide hilelinks esconde os links de terem bordas feias
\usepackage[hidelinks]{hyperref}




% CORES

%tratar das cores
\usepackage{xcolor}
\usepackage{pagecolor,lipsum}% http://ctan.org/pkg/{pagecolor,lipsum}

%meter o fundo em pretos e os caracters em branco
\pagecolor{black}
\color{white}

%defenicao de uma cor
% \definecolor{name}{model}{color-spec}
\definecolor{aquamarine}{rgb}{0, 181, 190}





%MACROS

\def\eq1{y=\frac{x}{3x^2+x+1}}
\def\labelaxes{Remember to include a scale and label your axes.}




%comand for links
%ele recebe 2 argumentos, o primeiro #1 é o link e o segundo #2 é a descrição e como o link vai ser demostrado
\newcommand{\lk}[2]{\href{#1}{\textcolor{aquamarine}{#2}}}
\newcommand{\smol}{\small}
%este pacote faz com que as margens fiquem menores
\usepackage{fullpage}

%define os limites das margens
% \usepackage[top=2cm, bottom=2cm, left=1cm, right=1cm]{geometry}

%margin define as margens todas, paperwidth define o comprimento e o paperheight a altura do papel
% \usepackage[margin=1cm, paperwidth=20cm, paperheight=30cm]{geometry}
%opening
\title{Tutorial de \LaTeX{}}
\author{OnikenX}
%o que esta comentado é o que ele faz de default
% \date{\today}


\begin{document}

\maketitle

\tableofcontents

\newpage
\section{Introdução}
Isto é ficheiro em que se realiza o que esta no tutorial que se pode encontrar neste link para o youtube:\\
\url{https://www.youtube.com/playlist?list=PL1D4EAB31D3EBC449}

E um pequeno exemplo:


Suppose we are given a recangle with side lengths $x+1$ and $x+3$. Then the equantion $$A=x^2+5x+3$$ represents the area of the rectangle.
\newpage
\section{Notações Comuns de Matemática}
\subsection{superscripts:} 
$$2x^{34}$$
$$2x^{3x+4}$$
$$2x^{3x^4+5}$$

\subsection{subscripts:}
$$x_1$$
$$x_{12}$$
$${x_{x_2}}_2$$

\subsection{greek letters:}
$$\pi$$
$$\alpha$$
$$A*\pi r^2$$

\subsection{trig functions:}
$$y=\sin{x}$$
$$y=\cos{x}$$
$$y=\tan{x}$$

\subsection{lag functions:}
$$\log_{10}{x}$$
$$\ln{x}$$

\subsection{square roots:}
$$\sqrt{2}$$
$$\sqrt[3]{5}$$
$$\sqrt{x^2+y^2}$$
$$\sqrt{1+\sqrt{x}}$$

\subsection{Fractions:}

About $\displaystyle{\frac{2}{3}}$ of the glass is full.

$$\frac{x}{x^2+x+1}$$

$$\frac{\sqrt{x+1}}{\sqrt[2]{x-1}}$$

$$\frac{1}{1+\frac{1}{x}}$$

$$\sqrt{\frac{x}{x^2+x+1}}$$

\newpage
\section{Bracket Tables \& Arrays}
\subsection{The basics:}
$$(x+1)$$
$$3[2+(x+1)]$$
$$\{a,b,c\}$$
$$\$12.55$$

\subsection{Tamanhos adaptativos:}
$$3\left(\frac{2}{5}\right)$$
$$3\left[\frac{2}{5}\right]$$
$$3\left\{\frac{2}{5}\right\}$$

$$\left|\frac{x+1}{x}\right|$$

$$\left\{ x+2\right.$$

\subsection{Tabular:}
\begin{tabular}{|c|ccccc|}

\hline
 $x$ & 1 & 2 & 3 & 4 & 5  \\\hline
 $f(x)$ & 10 & 11 & 12 & 13 & 14\\ \hline
 
 \end{tabular}

 \subsection{Equation:}
 
 \begin{eqnarray}% para esconder os numeros mete se * a frente do eqnarray assim eqnarray* e os numeros desaparecem
  5x^2-9&=&x+3\\%para meter os iguais no meio é preciso meter los entre &
  4x^2&=&12\\
  x^3&=&3\\
  x&\approx&\pm1.732
 \end{eqnarray}

 \subsection{List:}
 %As listas enumeradas automaiticamente se identificam
 \begin{enumerate}
  \item Escreva x num espaço de x.
  \item paper
  \begin{enumerate}
   \item assessments
   \begin {enumerate}
   \item Boas
   \end {enumerate}
   \item homework
   \item notes
  \end{enumerate}
 \end{enumerate}  

 %uma lista de item nao tem organizacao
 \subsection{Itens:}
 \begin{itemize}
  \item Escreva x num espaço de x.
  \item paper
  \begin{itemize}
   \item assessments
   \begin {itemize}
   \item Boas
   \end {itemize}
   \item homework
   \item notes
  \end{itemize}
 \end{itemize}  

 %para meter mos as labels que quizermos mete se
 \begin{enumerate}
  \item[Commutative]$a+b=b+a$
  \item[Associative]$a+(b+c)=(a+b)$
  \item[Distributive]$a+(b+c)=ab+ac$
 \end{enumerate}
 
 \newpage
\section{Texto e formatação de documentos}
\subsection{Italico, negrito e typewriter:}
Isto ira produzir \textit{texto italico}.
Este irá produzir \textbf{texto em negrito}.
Isto ira produzir o texto em fonte \texttt{typewriter}.

Porfavor visitem o meu website: 
% \begin{textcolorblue}
% \textit{\href{https://onikenx.github.io}{onikenx.github.io}}.
% \end{}
\subsection{Tamanhos e alinhamentos:}
\begin{center}

Porfavor desculpem a minha tia anabela.\\



\tiny
Porfavor desculpem a minha tia anabela.

\smol
Porfavor desculpem a minha tia anabela.

\normalsize
Porfavor desculpem a minha tia anabela.

\large
Porfavor desculpem a minha tia anabela.

\Large
Porfavor desculpem a minha tia anabela.

\huge
Porfavor desculpem a minha tia anabela.

\Huge
Porfavor desculpem a minha tia anabela.

\end{center}

\begin{flushright}

 This is a right align text.
\end{flushright}

\begin{flushleft}
 This is a left align text.
\end{flushleft}

$\rightarrow$
$\leftarrow$

Aqui tem o \lk{http://www.sascha-frank.com/Arrow/latex-arrows.html}{link} as para varias setas que se pode fazer em \LaTeX{}.

\section{Packages, Macros \& Graphics}
\subsection{Pacotes:}
    Pacotes servem para expandir as funcionaridades basicas do \LaTeX{}, exemplos podem ser encontrados na pagina fonte que se pode encontrar \href{https://github.com/OnikenX/LaTeX-showcase}{\textcolor{aquamarine}{neste repositorio no github}}.
    
    
    Teste do comando lk:
    \lk{http://ecasia.com}{ecasia website}

\subsubsection{Denotações matemáticas:}    
\par\indent    
    O conjunto de números naturais denomina-se
    $\mathbb{N}$
    .
    
    O conjunto de números inteiros denomina-se
    $\mathbb{Z}$
    .
    
    O conjunto de números racionais denomina-se
    $\mathbb{Q}$
    .
    
    O conjunto de números reais denomina-se
    $\mathbb{R}$
    .
    
    \subsection{Macros:}
    
    Graph $\eq1$. \labelaxes
    
    Identify the asymptotes for the graph of $\eq1$.
    
    
    
    
\end{document}
