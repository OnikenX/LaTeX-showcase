\documentclass[a4paper,10pt]{article}
\usepackage[utf8]{inputenc}
\usepackage{xspace}
\usepackage[portuguese]{babel}
\usepackage{hyperref}

%opening
\title{Tutorial de \LaTeX\xspace}
\author{OnikenX}

\begin{document}

\maketitle

\newpage
\section{Introdução}
Isto é ficheiro em que se realiza o que esta no tutorial que se pode encontrar neste link para o youtube:\\
\url{https://www.youtube.com/playlist?list=PL1D4EAB31D3EBC449}

E um pequeno exemplo:


Suppose we are given a recangle with side lengths $x+1$ and $x+3$. Then the equantion $$A=x^2+5x+3$$ represents the area of the rectangle.
\newpage
\section{Notações Comuns de Matemática}
superscripts: 
$$2x^{34}$$
$$2x^{3x+4}$$
$$2x^{3x^4+5}$$
\\
subscripts:
$$x_1$$
$$x_{12}$$
$${x_{x_2}}_2$$
\\
greek letters:
$$\pi$$
$$\alpha$$
$$A*\pi r^2$$
\\
trig functions:
$$y=\sin{x}$$
$$y=\cos{x}$$
$$y=\tan{x}$$
\\
lag functions:
$$\log_{10}{x}$$
$$\ln{x}$$
\\
square roots:
$$\sqrt{2}$$
$$\sqrt[3]{5}$$
$$\sqrt{x^2+y^2}$$
$$\sqrt{1+\sqrt{x}}$$
\\
Fractions:

About $\displaystyle{\frac{2}{3}}$ of the glass is full.

$$\frac{x}{x^2+x+1}$$

$$\frac{\sqrt{x+1}}{\sqrt[2]{x-1}}$$

$$\frac{1}{1+\frac{1}{x}}$$

$$\sqrt{\frac{x}{x^2+x+1}}$$

\newpage
\section{Bracket Tables \& Arrays}
\huge
$$(x+1)$$
$$3[2+(x+1)]$$
$$\{a,b,c\}$$
$$\$12.55$$

$$3\left(\frac{2}{5}\right)$$
$$3\left[\frac{2}{5}\right]$$
$$3\left\{\frac{2}{5}\right\}$$

$$\left|\frac{x+1}{x}\right|$$

$$\left\{ x+2\right.$$

Tabular:\\
\begin{tabular}{|c|ccccc|}

\hline
 $x$ & 1 & 2 & 3 & 4 & 5  \\\hline
 $f(x)$ & 10 & 11 & 12 & 13 & 14\\ \hline
 
 \end{tabular}

 \newpage
 Equation:
 
 \begin{eqnarray}% para esconder os numeros mete se * a frente do eqnarray assim eqnarray* e os numeros desaparecem
  5x^2-9&=&x+3\\%para meter os iguais no meio é preciso meter los entre &
  4x^2&=&12\\
  x^3&=&3\\
  x&\approx&\pm1.732
 \end{eqnarray}

 List:
 %As listas enumeradas automaiticamente se identificam
 \begin{enumerate}
  \item Escreva x num espaço de x.
  \item paper
  \begin{enumerate}
   \item assessments
   \begin {enumerate}
   \item Boas
   \end {enumerate}
   \item homework
   \item notes
  \end{enumerate}
 \end{enumerate}  

 %uma lista de item nao tem organizacao
 Itens:
 \begin{itemize}
  \item Escreva x num espaço de x.
  \item paper
  \begin{itemize}
   \item assessments
   \begin {itemize}
   \item Boas
   \end {itemize}
   \item homework
   \item notes
  \end{itemize}
 \end{itemize}  

 \begin{enumerate}
  \item $a+b=b+a$
  \item$a+(b+c)=(a+b)$
  \item$a+(b+c)=ab+ac$
 \end{enumerate}

 
\end{document}
