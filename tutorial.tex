\documentclass[a4paper,10pt]{article}
\usepackage[utf8]{inputenc}
\usepackage{xspace}
\usepackage[portuguese]{babel}
\usepackage{hyperref}

%opening
\title{Tutorial de \LaTeX{}}
\author{OnikenX}
%isto estar comentado ou nao faz diferença
% \date{\today}
\begin{document}

\maketitle

\tableofcontents
\newpage
\section{Introdução}
Isto é ficheiro em que se realiza o que esta no tutorial que se pode encontrar neste link para o youtube:\\
\url{https://www.youtube.com/playlist?list=PL1D4EAB31D3EBC449}

E um pequeno exemplo:


Suppose we are given a recangle with side lengths $x+1$ and $x+3$. Then the equantion $$A=x^2+5x+3$$ represents the area of the rectangle.
\newpage
\section{Notações Comuns de Matemática}
\subsection{superscripts:} 
$$2x^{34}$$
$$2x^{3x+4}$$
$$2x^{3x^4+5}$$

\subsection{subscripts:}
$$x_1$$
$$x_{12}$$
$${x_{x_2}}_2$$

\subsection{greek letters:}
$$\pi$$
$$\alpha$$
$$A*\pi r^2$$

\subsection{trig functions:}
$$y=\sin{x}$$
$$y=\cos{x}$$
$$y=\tan{x}$$

\subsection{lag functions:}
$$\log_{10}{x}$$
$$\ln{x}$$

\subsection{square roots:}
$$\sqrt{2}$$
$$\sqrt[3]{5}$$
$$\sqrt{x^2+y^2}$$
$$\sqrt{1+\sqrt{x}}$$

\subsection{Fractions:}

About $\displaystyle{\frac{2}{3}}$ of the glass is full.

$$\frac{x}{x^2+x+1}$$

$$\frac{\sqrt{x+1}}{\sqrt[2]{x-1}}$$

$$\frac{1}{1+\frac{1}{x}}$$

$$\sqrt{\frac{x}{x^2+x+1}}$$

\newpage
\section{Bracket Tables \& Arrays}
\subsection{The basics:}
$$(x+1)$$
$$3[2+(x+1)]$$
$$\{a,b,c\}$$
$$\$12.55$$

\subsection{Tamanhos adaptativos:}
$$3\left(\frac{2}{5}\right)$$
$$3\left[\frac{2}{5}\right]$$
$$3\left\{\frac{2}{5}\right\}$$

$$\left|\frac{x+1}{x}\right|$$

$$\left\{ x+2\right.$$

\subsection{Tabular:}
\begin{tabular}{|c|ccccc|}

\hline
 $x$ & 1 & 2 & 3 & 4 & 5  \\\hline
 $f(x)$ & 10 & 11 & 12 & 13 & 14\\ \hline
 
 \end{tabular}

 \subsection{Equation:}
 
 \begin{eqnarray}% para esconder os numeros mete se * a frente do eqnarray assim eqnarray* e os numeros desaparecem
  5x^2-9&=&x+3\\%para meter os iguais no meio é preciso meter los entre &
  4x^2&=&12\\
  x^3&=&3\\
  x&\approx&\pm1.732
 \end{eqnarray}

 List:
 %As listas enumeradas automaiticamente se identificam
 \begin{enumerate}
  \item Escreva x num espaço de x.
  \item paper
  \begin{enumerate}
   \item assessments
   \begin {enumerate}
   \item Boas
   \end {enumerate}
   \item homework
   \item notes
  \end{enumerate}
 \end{enumerate}  

 %uma lista de item nao tem organizacao
 Itens:
 \begin{itemize}
  \item Escreva x num espaço de x.
  \item paper
  \begin{itemize}
   \item assessments
   \begin {itemize}
   \item Boas
   \end {itemize}
   \item homework
   \item notes
  \end{itemize}
 \end{itemize}  

 %para meter mos as labels que quizermos mete se
 \begin{enumerate}
  \item[Commutative]$a+b=b+a$
  \item[Associative]$a+(b+c)=(a+b)$
  \item[Distributive]$a+(b+c)=ab+ac$
  
 \end{enumerate}
 
 \newpage
\section{Texto e formatação de documentos}
Isto ira produzir \textit{texto italico}.


Este irá produzir \textbf{texto em negrito}.

Isto ira produzir o texto em fonte \texttt{typewriter}.


Porfavor visitem o meu website \texttt{onikenx.github.io}.


\begin{center}

Porfavor desculpem a minha tia anabela.

\tiny
Porfavor desculpem a minha tia anabela.

\small
Porfavor desculpem a minha tia anabela.

\normalsize
Porfavor desculpem a minha tia anabela.

\large
Porfavor desculpem a minha tia anabela.

\Large
Porfavor desculpem a minha tia anabela.

\huge
Porfavor desculpem a minha tia anabela.

\Huge
Porfavor desculpem a minha tia anabela.

\end{center}

\begin{flushright}

 This is a right align text.
\end{flushright}

\begin{flushleft}
 This is a left align text.
\end{flushleft}



$\rightarrow$
$\leftarrow$

\end{document}
